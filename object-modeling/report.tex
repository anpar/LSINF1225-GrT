\input{../lib.tex}
\doctitle{Rapport : modélisation des objets}


\section{Diagramme de classes UML}
\begin{itemize}
	\item[{$\bullet$}] \textbf{User} : cette classe a toutes les fonctionnalités de base que requièrent
	tous les types d'utilisateurs confondus.
	\item[{$\bullet$}] \textbf{Customer} : cette classe représente le client du bar qui est connecté.
	Elle contient tous les éléments de la classe "User" avec quelques méthodes/attributs qui lui sont utiles
	en tant que client connecté.
	\item[{$\bullet$}] \textbf{Waiter} : cette classe représente un serveur du bar. Cette classe a les
	fonctionnalités de base de "User" mais possède ses propres méthodes/attributs pour modifier et créer
	des commandes ainsi que des additions.
	\item[{$\bullet$}] \textbf{Drink} : cette classe représente les boissons de la base de données avec
	des méthodes/attributs utiles. 
	\item[{$\bullet$}] \textbf{MenuDisplayer} : cette classe permet d'afficher les
	différentes boissons et de rechercher des boissons en fonction de l'utilisateur. Cette classe a
	absolument besoin de la classe "Drink".
	\item[{$\bullet$}] \textbf{Order} : cette classe représente une commande faite par un serveur. La
	classe a impérativement besoin de la classe "Drink" pour pouvoir fonctionner. Elle est utilisée par
	le serveur qui peut créer et modifier une commande.
	\item[{$\bullet$}] \textbf{Bill} : cette classe représente une addition. Elle a besoin de
	la classe "Order" pour faire des opérations comme calculer le total d'une ou plusieurs
	commandes, sur une collection de commandes. Elle est utilisée par le serveur qui peut, à tout moment,
	demander l'addition de x commandes.
\end{itemize}
\section{Diagrammes de séquences UML}
\subsection{Afficher la carte pour un client et chercher des boissons}
Lorsqu'un utilisateur se connecte à l'application, il initialise un ``afficheur
de carte'' (c'est à dire un objet de type MenuDisplayer) adapté au type
d'utilisateur (client ou serveur) et dont tous les critères de recherches
sont initialisés à une valeur par défaut. Cet opération se fait via la méthode
initMenuDisplayer() de la classe User. Celle-ci retourne un objet de type
 MenuDisplayer qui peut ensuite être utilisé pour afficher et effectuer
des recherches dans la carte.

Lorsqu'un client désire afficher la carte, il utilise une méthode de la classe "MenuDisplayer" appelée display(). Cette méthode va chercher toutes les informations nécessaires sur les boissons, en utilisant des méthodes de la classe "Drink" sur une collection d'objets "Drink", pour ensuite les afficher. 

La méthode suivante, setCriterion(), n'apparait pas dans notre diagramme de classe car ce n'est pas une méthode à proprement parler. Elle représente les différents "setter" qui vont être utilisés pour initialiser les attributs des filtres de MenuDisplayer. 

Lorsque le client veut chercher une boisson, il initialise ses critères de recherche (setCriterion()), un display() est de nouveau utilisé et MenuDisplayer recherche des informations sur les boissons nécessaires pour les afficher. 

\subsection{Ajout de boissons à une commande}

Lorsqu'un serveur désire ajouter une boisson à une commande, la démarche de recherche de la boisson est identique à celle du client. Une fois la boisson trouvée, il l'ajoute à l'objet "Order". Ensuite, si la commande est terminée, il utilise la méthode close() de la classe "Order" pour terminer la commande.


\subsection{Initialisation d'une addition par un serveur}
Lorsqu'un serveur veut initialiser une addition, il utilise la méthode print() de la classe "Bill" sur un objet bill. L'objet bill va alors rechercher toutes les commandes dont l'addition est composée. Il va ensuite chercher toutes les informations sur les boissons dont les commandes sont composées.

Une fois l'addition composée et payée, le serveur utilise la méthode close() de l'objet bill pour supprimer toutes les commandes liées à cette addition. Une fois cela fait, un message "billErase" est retourné.

\subsection{Ajout d'une note à une boisson par un client} 

Lorsqu'un client désire noter une boisson, il recherche la boisson de la même manière que précédemment. Une fois la boisson (drink 1) trouvée, l'objet utilise la méthode AddAvis() de la classe "Client".

% TODO

\input{../footer.tex}
